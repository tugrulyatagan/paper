\documentclass[conference]{IEEEtran}
\IEEEoverridecommandlockouts
\usepackage{cite}
\usepackage{amsmath,amssymb,amsfonts}
\usepackage{algorithmic}
\usepackage{graphicx}
\usepackage{textcomp}
\usepackage{xcolor}
\def\BibTeX{{\rm B\kern-.05em{\sc i\kern-.025em b}\kern-.08em
    T\kern-.1667em\lower.7ex\hbox{E}\kern-.125emX}}
\begin{document}

\title{Paper Title*
}


\author{\IEEEauthorblockN{Tugrul Yatagan\IEEEauthorrefmark{1} and
Sema F. Oktug\IEEEauthorrefmark{2}}
\IEEEauthorblockA{Department of Computer Engineering,
Istanbul Technical University\\
Istanbul, Turkey\\
Email: \IEEEauthorrefmark{1}yatagan@itu.edu.tr,
\IEEEauthorrefmark{2}oktug@itu.edu.tr}}
\maketitle


\begin{abstract}
Low power wide area network (LPWAN) technologies offer affordable connectivity to massive number of low-power devices distributed over very large geographical areas using license-free frequency bands. Focus on this work is one of the most promising LPWAN technologies: Lora. LoRa offers long range communication and strong resilience to interference by proprietary modulation technique based on Chirp Spread Spectrum (CSS). LoRa trades data rate for sensitivity and communication range by spreading symbols within a fixed channel bandwidth. SF assignment of nodes has strong effect on collisions thus network performance. In this work, we implemented a simulation environment to study different SF assignment schemes and we proposed a machine learning technique to optimize SF assignment. Finally, we illustrate simulation results for proposed machine learning SF assignment technique.
\end{abstract}


\begin{IEEEkeywords}
LoRa, Spreading Factor, IoT, LPWAN, Machine Learning
\end{IEEEkeywords}


\section{Introduction}
\par In the last few years, number of Internet of Things applications increase exponentially. \cite{7721743} Recent developments on LPWAN technologies has great impact on IoT application number growth. LPWAN technologies offer solutions to some of the oldest wireless communication challenges. Traditional wireless communication methods such as cellular networks (e.g., 2G, 3G, LTE) and short-range communication methods (e.g., Bluetooth, WiFi, Zigbee) cannot provide low power and long range at the same time. (CITE) Cellular networks can provide long range and high data rate but they are complex and consume too much power. Besides, most of the IoT applications don't require high data rate. Short-range communication methods can provide relatively low power consumption but their range is limited to a few hundred meters at best. \cite{7815384} LPWAN technologies fill the technology gap between short range and cellular communication by providing low power and long range communication. LPWAN technologies basically sacrifice data rate to provide low power consumption.

\par There are several emerging LPWAN technologies. LoRa, SigFox, NB-IoT and LTE-M are commonly used, well known LPWAN technologies. LoRa and SigFox use license free ISM frequency bands while NB-IoT and LTE-M use licensed frequency bands which brings extra cost. (CITE) Both LoRa and SigFox known for ultra low power consumption and resilient to interference while NB-IoT and LTE-M are promoted for higher data rate. LoRa has open standard MAC protocol called LoRaWAN. (CITE) LoRaWAN and SigFox MAC protocols are based on pure ALOHA medium access. (CITE) LoRaWAN networks can be deployed as private network. However, SigFox and NB-IoT supports only operator contracted deployments. (CITE) Number of messages that Sigfox end device can send in a day is limited to 140 packets for uplink and just 4 packets for downlink. Sigfox packet payload is limited to 12 bytes for uplink and 8 bytes for downlink. (CITE) However LoRa supports up to 243 bytes payload and NB-IoT supports up to 1600 bytes payload. (CITE)
Sigfox maximum data rate is also very low 100 bps, LoRa 50 kbps, NB-IoT 200 kbps.\\
LoRa can change data rate by spreading symbols within a fixed channel bandwidth. This enables tradeoff between sensitivity and data rate. LoRa has 7 different spreading factor option. Different spreading factor communications are orthogonal to each other.

\par TODO: SF Assignment Issue on LoRa

\par TODO: Proposal for SF Assignment Issue on LoRa

\par TODO: Organization of the paper


\section{LoRa}
\par LoRa is the name of the physical layer Radio/Chipset technology that provides wireless link for low power wide area networks. LoRa uses proprietary spread spectrum modulation technique that is derivative of Chirp Spread Spectrum (CSS). LoRa is registered trademark of Semtech Corporation. LoRa has an open standard MAC protocol called LoRaWAN. LoRa and LoRaWAN terms are frequently and incorrectly used for each other.

\subsection{LoRaWAN}
\par LoRaWAN is a media access control (MAC) layer protocol which designed for large scale LoRa networks. LoRaWAN is an open source standard developed and maintained by LoRa Alliance. LoRa Alliance is an open, non-profit organization dedicated to promoting the interoperability and standardization of LoRaWAN. LoRa can be used without LoRaWAN as a wireless link technology, however LoRaWAN is developed considering well known LPWAN challenges and their best practice solutions. LoRaWAN also provides interoperability between different networks. LoRaWAN uses pure ALOHA to access medium. LoRaWAN provides lightweight but powerful standard for IoT applications.

\par TODO: explain end node, gateway, class A

\subsection{LoRa Modulation}
\par TODO

\subsection{Spreading Factor}
\par LoRa CSS modulation can spread the symbols.\\
The spread factor is the ratio between symbol rate and chip rate. The ratio between symbol and chip rate is $2$\textsuperscript{spreadfactor}.\\
Six different spread factors are available (between 7 and 12); increasing the spread factor makes the signal more robust to noise, but decreases the data rate.\\

\par Lowest SF suitable area section is relatively higher then the others. Most of the end devices which close to GW will fall into this area.

\par End devices near the GW will probably select lowest SF all the time. Which results a lot of collisions between lowest SF transmissions while end device number near the GW increases. Figure~\ref{fig:collision}

\begin{figure}
\centering
\includegraphics[width=0.5\textwidth]{collision}
\caption{Collision between nodes near the GW.}
\label{fig:collision}
\end{figure}


\section{Other Related Works}
The literature related to the work presented in this paper has started growing recently. LPWAN technologies and especially LoRa/LoRaWAN attract researchers attention lately. We listed some of these works which studying  LoRaWAN and LoRa spreading factor.

\par In \cite{7996384}, the authors evaluated the performance of LoRa networks in a smart city scenario. The
authors proposed a link measurement and a link performance model for LoRa. The authors also proposed a SINR threshold matrix for modeling LoRa interference between simultaneous but different spreading factor LoRa transmissions. They implement a LoRa simulator in ns-3 to study scalability and performance of LoRaWAN networks. Their results show that LoRaWAN networks scale well as the number of nodes and gateways increases. They also show that spreading factor assignment has great effect on network performance.

\par In \cite{8090518}, another LoRaWAN ns-3 simulator is presented. Authors introduced an error model for determining range as well as interference between multiple simultaneous LoRa transmissions. Their simulator supports LoRaWAN class A end devices, multiple gateways, both upstream and downstream confirmed messages. Their results show that allocating network parameters to end devices is hugely important for the performance of LoRaWAN networks.

\par In \cite{s17061193}, the authors investigated single gateway LoRaWAN network scalability in terms of the number of end nodes using a simulation model based on real measurements. They measure the impact of two concurrent LoRa transmissions on each other by using physical LoRaWAN end devices and a gateway then they created a simulation model from measurements. Their results show that LoRaWAN has better scalability than pure ALOHA since a LoRa packet may still go through under collision if the the last six symbols of preamble and header of the packet does not collide.

\par In \cite{8267219}, the authors studied imperfect orthogonality between different LoRa spreading factor transmissions. The authors state that a LoRa transmission can be interfered even between different spreading factor transmissions when power of the interfering signal significantly overcomes the reference signal. Their experimental results show that this power difference is around 16 dB and this power difference can be seen when  interferer is close to receiver or multiple interferer can create this power difference cumulatively. 

\par In \cite{8430542}, the authors investigated the impact of interference caused by simultaneous LoRa transmissions using the same spreading factor as well as different spreading factors. They derived aggregated co-SF and inter-SF interference power SIR distributions to capture the coverage performance with respect to the distance from the gateway for modeling interference in multiple gateway scenarios. Their results show that transmission among different spreading factors can cause a significant impact in high-density LoRaWAN networks.

\section{Proposed Technique}
\par If GW force some of close end nodes to select higher SF even if they can able to communicate with lower SF, this will result lower collisions due to the orthogonality of different SF. Figure~\ref{fig:collision_solution_single_gw}

\par GW can avoid increasing SF of end nodes which is close to other GW using location information obtained by triangulating the signal strength of end nodes. Figure~\ref{fig:collision_solution_multi_gw}

\begin{figure}
\centering
\includegraphics[width=0.5\textwidth]{collision_solution_single_gw}
\caption{Collision between nodes near the GW can be prevented by using higher SF.}
\label{fig:collision_solution_single_gw}
\end{figure}

\begin{figure}
\centering
\includegraphics[width=0.5\textwidth]{collision_solution_multi_gw}
\caption{Collision prevention proposal for intersecting GWs.}
\label{fig:collision_solution_multi_gw}
\end{figure}


\section{Simulation Environment and Results}
A discrete event simulator is implemented in Python to simulate LoRa network performance. 
https://github.com/tugrulyatagan/simlorafs

\subsection{Simulation Environment}

\subsubsection{Link Model}
TODO

\subsubsection{Interference Model}
TODO

\subsubsection{Simulation Assumptions}
TODO

\begin{figure}
\centering
\includegraphics[width=0.5\textwidth]{sf_pdr}
\caption{PDR of different spreading factors.}
\label{fig:sf_1000}
\end{figure}

\begin{figure}
\centering
\includegraphics[width=0.5\textwidth]{gw_pdr}
\caption{PDR of different gateway numbers.}
\label{fig:sf_3000}
\end{figure}

\begin{figure}
\centering
\includegraphics[width=0.5\textwidth]{r_pdr}
\caption{PDR of different topology radius.}
\label{fig:sf_10000}
\end{figure}

\begin{figure}
\centering
\includegraphics[width=0.5\textwidth]{pr_pdr}
\caption{PDR of different packet rates.}
\label{fig:sf_10000}
\end{figure}

\subsection{Simulation Results}
TODO: Evaluation


\section{Conclusion}
TODO: Summary, future works
\cite{7815384} \cite{7803607} \cite{7996384} \cite{8090518} \cite{s17061193} \cite{8267219} \cite{8430542} \cite{8319183} \cite{8480649} \cite{AN1200.22} \cite{Bor:2016:LLW:2988287.2989163} \cite{8406255} \cite{DBLP:journals/corr/abs-1802-10338} \cite{finnegan2018comparative}


\section*{Acknowledgment}
This work is supported by Turkish Ministry of Development and Istanbul Technical University researcher support program under the Grant No. ITU-AYP-2017-1.


\bibliographystyle{IEEEtran}
\bibliography{references}


\end{document}
